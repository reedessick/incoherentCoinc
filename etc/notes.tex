\documentclass{article}

%-------------------------------------------------

\usepackage{fullpage}

\usepackage{multirow}

\usepackage{amsmath}
\usepackage{amsfonts}
\usepackage{amssymb}

\usepackage{color}

%-------------------------------------------------
\begin{document}
%-------------------------------------------------

\section*{basics of Poisson distribution}
\label{sec:poisson}

The Poisson distribution is defined over the counting numbers ($n$)
\begin{equation}
    p(n|\mu) = \frac{\mu^n}{n!}e^{-\mu}
\end{equation}
with corresponding moment generating function
\begin{equation}
    \left<e^{sn}\right> = \sum\limits_{n=0}^{n=\infty} e^{sn} p(n|\mu) = e^{\mu(e^s - 1)}
\end{equation}
so that
\begin{align}
    \left<n\right> & = \mu \\
    \left<n^2\right> & = \mu^2 + \mu \\
    \left<n^3\right> & = \mu^3 + 3\mu^2 + \mu \\
    \left<n^4\right> & = \mu^4 + 6\mu^3 + 6\mu^2 + \mu 
\end{align}

%-------------------------------------------------

\section*{definition of statitics}
\label{sec:statistics}

We divide the experiment into a sequence of $N$ independent trials, each of duration $\tau$.
We consider a model with independent (uncorrelated) processes in two detectors ($A$ and $B$) and a third process that appears in both detectors.
Within each segments ($i$), we then have
\begin{align}
    a_i & \sim \textrm{Poisson}(\mu_a = \tau\lambda_a) \\
    b_i & \sim \textrm{Poisson}(\mu_b = \tau\lambda_b) \\
    s_i & \sim \textrm{Poisson}(\mu_s = \tau\lambda_s)
\end{align}

We also consider larger coincidence windows in a dyadic hierarchy ($\tau^{(n)} = 2^n \tau$) with $M = N/2^n$ segments.
With can then define

\noindent
count in detector A
\begin{equation}
    c_A^{(n)} = \sum\limits_x^M \sum\limits_i^{2^n} (a_{2^n x + i} + s_{2^n x + i})
\end{equation}

\noindent
count in detector B
\begin{equation}
    c_B^{(n)} = \sum\limits_x^M \sum\limits_i^{2^n} (b_{2^n x + i} + s_{2^n x + i})
\end{equation}

\noindent
coincidences in zero-lag
\begin{equation}
    c_0^{(n)} = \sum\limits_x^M \left[\sum\limits_i^{2^n} (a_{2^n x + i} + s_{2^n x + i})\right] \left[\sum\limits_j^{2^n} (a_{2^n x + j} + s_{2^n x + j})\right]
\end{equation}

\noindent
coincidences from time-slides retaining zero-lag
\begin{equation}
    c_+^{(n)} = \sum\limits_x^M \left[ \sum\limits_i^{2^n} (a_{2^n x + i} + s_{2^n x + i})\right] \sum\limits_y^M \left[ \sum\limits_j^{2^n}(b_{2^n y + j} + s_{2^n y + j})\right]
\end{equation}

\noindent
coincidences from time-slides after removing zero-lag
\begin{equation}
    c_{-}^{(n)} = \sum\limits_x^M \left[ \delta\left(\sum\limits_i^{2^n} (b_{2^n x + i} + s_{2^n x + i}) \right) \sum\limits_j^{2^n} (a_{2^n x + j} + s_{2^n x + j})\right] \sum\limits_y^M \left[ \delta\left(\sum \limits_p^{2^n} (a_{2^n y + p} + s_{2^n y + p})\right) \sum\limits_q^{2^n}(b_{2^n y + q} + s_{2^n y + q})\right]
\end{equation}
where
\begin{equation}
    \delta(x) = \left\{ \begin{matrix} 1 & x = 0 \\ 0 & x > 0 \end{matrix} \right.
\end{equation}

We also note that a few of these statistics do not depend on $n$.
Specifically,
\begin{align}
    c_A & = c_A^{(n)} = \sum\limits_i^N (a_i + s_i) \\
    c_B & = c_B^{(n)} = \sum\limits_i^N (b_i + s_i) \\
    c_+ & = c_+^{(n)} = \sum\limits_i^N (a_i + s_i) \sum\limits_j^N (b_j + s_j)
\end{align}

%-------------------------------------------------

\section*{moments}
\label{sec:moments}

We compute moments of our statistics via
\begin{equation}
    \mathrm{E}[f] = \prod\limits_i^N\left[ \sum_{a_i}^\infty p(a_i) \sum_{b_i}^\infty p(b_i) \sum_{s_i}^\infty p(s_i) \right] f(a_1, \ldots, a_N, b_1, \ldots, b_N, s_1, \ldots, s_N)
\end{equation}
We then estimate the covariance of statistics via
\begin{equation}
    \mathrm{Cov}[fg] = \mathrm{E}[fg] - \mathrm{E}[f]\mathrm{E}[g]
\end{equation}
This then produces a basic Gaussian model for the likelihood, which should be reasonably accurate when the signal counts are high (the Poisson distribution approaches a Guassian as $\mu \rightarrow \infty$).

%------------------------

\subsection*{first moments}
\label{sec:first moments}

We compute the first moments of our statistics
\begin{align}
    \mathrm{E}[c_A]
        &= N(\mu_a + \mu_s) \\
    \mathrm{E}[c_B]
        & = N(\mu_b + \mu_s) \\
    \mathrm{E}[c_0^{(n)}]
        & = M [2^n (\mu_a + \mu_s)] [2^n (\mu_b + \mu_s)] + M 2^n \mu_s \nonumber \\
        & = N 2^n (\mu_a + \mu_s)(\mu_b + \mu_s) + N \mu_s \\
    \mathrm{E}[c_+]
        & = N \mu_s + N^2 (\mu_a + \mu_s) (\mu_b + \mu_s) \\
    \mathrm{E}[c_{-}^{(n)}]
         & = M (M-1) [2^n \mu_a] [2^n \mu_b] e^{-2^n (\mu_a + \mu_b + 2\mu_s)} \nonumber \\
         & = N (N - 2^n) \mu_a \mu_b e^{-2^n (\mu_a + \mu_b + 2\mu_s)}
\end{align}

%------------------------

\subsection*{second moments}
\label{sec:second moments}

We then compute the second moments of our statistics.
We report only a subset of possible combinations as the rest of the combinations can be derived from these expressions.
In what follows, we denote expectation values for individual windows (of size $\tau$) with $\left<\cdot\right>$.

%---

\begin{align}
    \mathrm{E}[c_A c_A]
        & = N \mu_a^2 + N \mu_a
\end{align}

%---

\begin{align}
    \mathrm{E}[c_A c_B]
        & = N \mu_s + N^2 (\mu_a + \mu_s)(\mu_b + \mu_s) \\
        & = E[c_+] \nonumber
\end{align}

%---

\begin{align}
    \mathrm{E}[c_A c_+]
        & = M 2^n \left<(a+s)^2 (b+s)^2\right> \nonumber \\
        & \quad + M 2^n (2^n-1) \left[ 2\left<a+s\right>\left<(a+s)(b+s)\right> + \left<b+s\right>\left<(a+s)^2\right>\right] \nonumber \\
        & \quad + M 2^n (2^n-1) (2^n-2) \left<a+s\right>^2 \left<b+s\right> \nonumber \\
        & \quad + M (M-1) 2^{n+1} \left<a+s\right> \left[ 2^n\left<(a+s)(b+s)\right> + 2^n(2^n-1)\left<a+s\right>\left<b+s\right>\right] \nonumber \\
        & \quad + M (M-1) 2^{n} \left<b+s\right> \left[ 2^n\left<(a+s)^2\right> + 2^n(2^n-1)\left<a+s\right>^2\right] \nonumber \\
        & \quad + M (M-1) (M-2) 2^{3n} \left<a+s\right>^2 \left<b+s\right>
\end{align}

%---

\begin{align}
    \mathrm{E}[c_A c_0^{(n)}]
        & = M 2^n \left<(a+s)^2(b+s)\right> \nonumber \\
        & \quad + M 2^n (2^n-1) \left[ 2\left<a+s\right>\left<(a+s)(b+s)\right> + \left<b+s\right>\left<(a+s)^2\right> \right] \nonumber \\
        & \quad + M 2^n (2^n-1) (2^n-2) \left<a+s\right>^2 \left<b+s\right> \nonumber \\
        & \quad + M (M-1) [2^n \left<a+s\right>] [2^n \left<(a+s)(b+s)\right> + 2^n (2^n-1) \left<a+s\right>\left<b+s\right>]
\end{align}

%---

\begin{align}
    \mathrm{E}[c_A c_{-}^{(n)}]
        & = M (M-1) e^{-2^n(\mu_a + \mu_s)} 2^n \left<b\right> e^{-2^n(\mu_b+\mu_s)} [ 2^n\left<a^2\right> + 2^n (2^n-1)\left<a\right>^2 ] \nonumber \\
        & \quad + M (M-1) (M-2) 2^n \left<a+s\right> 2^n \left<a\right> e^{-2^n(\mu_b+\mu_s)} 2^n \left<b\right> e^{-2^n(\mu_a+\mu_s)}
\end{align}

%---

\begin{align}
    \mathrm{E}[c_+ c_+]
        & = N \left<(a+s)^2 (b+s)^2\right> \nonumber \\
        & \quad + N (N-1) [ 2\left<a+s\right>\left<(a+s)(b+s)^2\right> + 2\left<b+s\right>\left<(a+s)^2(b+s)\right> ] \nonumber \\
        & \quad + N (N-1) (N-2) [ \left<a+s\right>\left<b+s\right>\left<(a+s)(b+s)\right> + \left<a+s\right>^2\left<(b+s)^2\right> \nonumber \\
        & \quad \quad \quad \quad + \left<a+s\right>\left<b+s\right>\left<(a+s)(b+s)\right> + \left<b+s\right>\left<a+s\right>\left<(a+s)(b+s)\right> \nonumber \\
        & \quad \quad \quad \quad + \left<b+s\right>^2\left<(a+s)^2\right> + \left<a+s\right>\left<b+s\right>\left<(a+s)(b+s)\right> ] \nonumber \\
        & \quad + N (N-1) (N-2) (N-3) \left<a+s\right>^2 \left<b+s\right>^2
\end{align}

%---

\begin{align}
    \mathrm{E}[c_+ c_0^{(n)}]
        & = M 2^n \left<(a+s)^2(b+s)^2\right> + 2^n (2^n-1) ( 2\left<a+s\right>\left<(a+s)(b+s)^2\right> + 2\left<b+s\right>\left<(a+s)^2(b+s)\right> ) \nonumber \\
        & \quad + M 2^n (2^n-1) (2^n-2) ( 4 \left<a+b\right>\left<b+s\right>\left<(a+s)(b+s)\right> + \left<a+s\right>^2\left<(b+s)^2\right> + \left<b+s\right>\left<(a+s)^2\right> ) \nonumber \\
        & \quad + M 2^n (2^n-1) (2^n-2) (2^n-3) \left<a+s\right>^2 \left<b+s\right>^2 \nonumber \\
        & \quad + M (M-1) 2^n \left<a+s\right> 2^n \left<(a+s)(b+s)^2\right> \nonumber \\
        & \quad + M (M-1) 2^n \left<a+s\right> 2^n(2^n-1)( 2\left<b+s\right>\left<(a+s)(b+s)\right> + \left<a+s\right>\left<(b+s)^2\right> ) \nonumber \\
        & \quad + M (M-1) 2^n \left<a+s\right> 2^n (2^n-1) (2^n-2) \left<a+s\right>\left<b+s\right>^2 ) \nonumber \\
        & \quad + M (M-1) 2^n \left<b+s\right> 2^n\left<(a+s)^2(b+s)\right> \nonumber \\
        & \quad + M (M-1) 2^n \left<b+s\right> 2^n(2^n-1) ( 2\left<a+s\right>\left<(a+s)(b+s)\right> + \left<b+s\right>\left<(a+s)^2\right>) \nonumber \\
        & \quad + M (M-1) 2^n \left<b+s\right> 2^n (2^n-1) (2^n-2) \left<a+s\right>^2\left<b+s\right> ) \nonumber \\
        & \quad + M (M-1) ( 2^n \left<(a+s)(b+s)\right> + 2^n(2^n-1)\left<a+s\right>\left<b+s\right>)^2 \nonumber \\
        & \quad + M (M-1) (M-2) [2^n \left<a+s\right>] [2^n\left<b+s\right>] [2^n\left<(a+s)(b+s)\right> + 2^n(2^n-1)\left<a+s\right>\left<b+s\right>]
\end{align}

%---

\begin{align}
    \mathrm{E}[c_+ c_{-}^{(n)}]
        & = M (M-1) (M-2) 2^n \left<a+s\right> 2^n \left<a\right> e^{-2^n\left<b+s\right>} e^{-2^n\left<a+s\right>}( 2^n\left<b^2\right> + 2^n(2^n-1)\left<b\right>^2 ) \nonumber \\
        & \quad + M (M-1) (M-2) 2^n \left<b+s\right> 2^n\left<b\right> e^{-2^n\left<a+s\right>} e^{-2^n\left<b+s\right>} ( 2^n\left<a^2\right> + 2^n(2^n-1)\left<a\right>^2 ) \nonumber \\
        & \quad + M (M-1) (M-2) e^{-2^n\left<b+s\right>} 2^n \left<a\right> e^{-2^n\left<a+s\right>} 2^n \left<b\right> ( 2^n\left<(a+s)(b+s)\right> + 2^n(2^n-1)\left<a+s\right>\left<b+s\right> ) \nonumber \\
        & \quad + M (M-1) (M-2) (M-3) 2^n \left<a+s\right> 2^n \left<b+s\right> e^{-2^n\left<b+s\right>} 2^n \left<a\right> e^{-2^n\left<a+s\right>} 2^n \left<b\right>
\end{align}

%---

\begin{align}
    \mathrm{E}[c_0^{(0)} c_0^{(n)}]
        & = M 2^n \left<(a+s)^2 (b+s)^2\right> \nonumber \\
        & \quad + M 2^n (2^n-1) [ \left<(a+s)(b+s)\right>\left<(a+s)(b+s)\right> + \left<(a+s)\right>\left<(a+s)(b+s)^2\right> + \left<b+s\right>\left<(a+s)^2(b+s)\right> ] \nonumber \\
        & \quad + M 2^n (2^n-1) (2^n-2) \left<(a+s)(b+s)\right>\left<a+s\right>\left<b+s\right> \nonumber \\
        & \quad + M (M-1) 2^n \left<(a+s)(b+s)\right> [ 2^n \left<(a+s)(b+s)\right> + 2^n (2^n-1) \left<a+s\right>\left<b+s\right> ]
\end{align}

%---

\begin{align}
    \mathrm{E}[c_0^{(0)} c_-^{(n)}]
        & = M (M-1) (M-2) [2^n \left<(a+s)(b+s)\right>] [2^n \left<a\right> e^{-2^n(\mu_b+\mu_s)}] [2^n \left<b\right> 2^{-2^n(\mu_a+\mu_s)}]
\end{align}

%---

\begin{align}
    \mathrm{E}[c_-^{(0)} c_0^{(n)}]
        & = M (M-1) (M-2) [2^n\left<(a+s)(b+s)\right> + 2^n(2^n-1)\left<a+s\right>\left<b+s\right>] [2^n \left<a\right> e^{-2^n\left<b+s\right>}] [2^n \left<b\right> e^{-2^n\left<a+s\right>}]
\end{align}

%---

\begin{align}
    \mathrm{E}[c_-^{(0)} c_-^{(n)}]
        & = M (M-1) (M-2) 2^n \left<a\right> e^{-\left<b+s\right>} 2^n \left<a\right> e^{-2^n\left<b+s\right>} ( 2^n\left<b^2\right> + 2^n(2^n-1)\left<b\right>^2 ) 2^{-2^n\left<a+s\right>} \nonumber \\
        & \quad + M (M-1) (M-2) 2^n \left<b\right> e^{-\left<a+s\right>} 2^n \left<b\right> e^{-2^n\left<a+s\right>} ( 2^n\left<a^2\right> + 2^n(2^n-1)\left<a\right>^2 ) e^{-2^n\left<b+s\right>} \nonumber \\
        & \quad + M (M-1) (M-2) 2^n \left<a\right> e^{-2^n\left<b+s\right>} 2^n \left<b\right>e^{-2^n\left<a+s\right>} 2^n (2^n-2) \left<a\right> e^{-\left<b+s\right>} \left<b\right> e^{-\left<a+s\right>} \nonumber \\
        & \quad + M (M-1) (M-2) (M-3) [2^n\left<a\right> e^{-\left<b+s\right>}] [2^n\left<b\right>e^{-\left<a+s\right>}] [2^n\left<a\right>e^{-2^n\left<b+s\right>}] [2^n\left<b\right> e^{-2^n\left<a+s\right>}]
\end{align}

%------------------------

\section*{sanity checks}
\label{sec:checks}

\textcolor{red}{
\begin{itemize}
    \item code up expressions
    \item confirm that covariance matrix is positive definite for any ($\mu_a$, $\mu_b$, $\mu_s$) and multiple values of $n$
    \item simulate data and confirm predictions match the observed moments; whether skewness affects our estimates significantly
    \item can we generalize this to the case where we do not perform all possible timeslides?
    \item how well can we determine background rates, signal rate? does this logic actually apply to searches?
\end{itemize}
}

%-------------------------------------------------
\end{document}
%-------------------------------------------------
